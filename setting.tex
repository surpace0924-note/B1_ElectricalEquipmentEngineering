\documentclass[autodetect-engine,dvipdfmx-if-dvi,ja=standard]{bxjsarticle}

% 二段組にするとき
% \documentclass[twocolumn,autodetect-engine,dvipdfmx-if-dvi,ja=standard]{bxjsarticle}

\setlength{\topmargin}{-0.3in}
\setlength{\oddsidemargin}{0pt}
\setlength{\evensidemargin}{0pt}
\setlength{\textheight}{46\baselineskip}
\usepackage{graphicx}     %図を表示するのに必要
\usepackage{color}        %jpgなどを表示するのに必要
% \usepackage[dvipdfmx]{graphicx}     %図を表示するのに必要
% \usepackage[dvipdfmx]{color}        %jpgなどを表示するのに必要
\usepackage{amsmath,amssymb}        %数学記号を出すのに必要
\usepackage{setspace}
\usepackage{eclclass}
\usepackage{cases}
\usepackage{here}
\usepackage{fancyhdr}

\pagestyle{fancy}
\rhead{\leftmark}
\lhead{\rightmark}

% 式の番号を(senction_num.num)のよう
\makeatletter
\@addtoreset{equation}{section}
\def\theequation{\thesection.\arabic{equation}}
\makeatother

% 画像の貼り付けを簡単にする
\newcommand{\pic}[1]
{
  \begin{figure}[H]
    \begin{center}
      \includegraphics[width=\textwidth/2]{#1}
    \end{center}
  \end{figure}
}