\section{直流発電機}
\subsection{直流発電機}
\begin{itemize}
  \item 電機子:磁束を切って起電力を誘導させる部分
  \item 界磁 :電機子に通じる磁束を作る部分
  \item ブラシ:整流子と接して,誘導起電力を取り出す
  \item 整流子:ブラシに接して誘導起電力を整流して直流にする部分
\end{itemize}

\subsubsection{電機子}
電機子鉄心,電機子巻線,整流子で構成される.

\subsubsection{電機子鉄心}
界磁とともに磁気回路を作る部分.
うず電流やヒステリシス現象による鉄損を生じる.
\begin{enumerate}
  \item 成層鉄心:鉄損を少なくするため,けい素鋼板を積み重ねた成層鉄心を用いる
  \item スロット:巻線を収めるための溝
  \item 歯   :スロットの間のこと
\end{enumerate}

\subsubsection{電機子巻線}
電機子の巻線,一般に亀甲型コイルを使用する.

\subsubsection{単相巻きと二層巻き}
スロット内にひとつのコイル片を納める方法を単相巻き,スロット内に二つを二層巻き.
\subsubsection{重ね巻と波巻}
鼓状巻きでは,重ね巻と波巻の二つの方法がある.
\paragraph*{重ね巻\\}
並列回路を作る.並列巻とも呼ばれる.\\
 極数=回路数\\
 低電圧・大電流
\paragraph*{波巻\\}
極数によらず二つの並列回路を作る.直列巻とも呼ばれる.\\
 高電圧・定電流
\subsubsection{整流子}
常にブラシと接しているため,丈夫に作られる.整流子片と絶縁片が交互に組み合わされている.
\subsubsection{ブラシ}
ブラシは中性軸ごとにおく必要がある.重ね巻ではブラシ数と極数が等しい.

\subsection{電機子反作用}
磁束の流れに偏りができること.
電機子反作用での軸ずれを考慮しない中性軸を幾何学的中性軸,考慮したものを電気的中性軸という.
\subsubsection{主磁束}
磁極の作る磁束
\subsubsection{交さ起磁力}
電機子だけに電流を流した場合
\subsubsection{偏磁作用}
主磁束と交さ起磁力を合わせたもの.
これが電機子反作用.
\subsubsection{補償巻線}
電機子反作用対策のひとつ.電機子電流と逆方向に電流を流し,電機子の起電力を打ち消す.

\subsection{誘導起電力}
電機子導体一本に誘導される起電力の平均は
\begin{eqnarray}
  e &=& vBl\, [\textrm{V}] \\
  v &=& \pi Dn\, [\textrm{m/s}] \\
  e' &=& Bl\pi Dn\, [\textrm{V}]
\end{eqnarray}


ブラシ間で得られる直流起電力$E$は
\begin{equation}
  \label{DC_Electromotiveforce}
  E = \frac{Z}{a} \cdot e = \frac{Z}{a} \cdot p \phi n\, [\textrm{V}]
\end{equation}
ただし.
\begin{subnumcases}
  {}
  e:\mbox{誘導起電力の平均}[\textrm{V}]\nonumber \\
  v:\mbox{周速度}[\textrm{m/s}]\nonumber \\
  B:\mbox{ギャップの平均磁束密度}[\textrm{T}]\nonumber \\
  l:\mbox{コイル片の有効長さ}[\textrm{m}]\nonumber \\
  D:\mbox{電機子直径}[\textrm{m}]\nonumber \\
  n:\mbox{回転速度}[\textrm{rps}]\nonumber \\
  a:\mbox{並列回路数}\nonumber \\
  Z:\mbox{電機子巻線の全導体数}\nonumber \\
  p:\mbox{極数}\nonumber \\
  \phi:\mbox{1極から出る磁束}[\textrm{Wb}]\nonumber
\end{subnumcases}

\subsection{直流発電機の種類}
主に他例発電機,分巻発電機,直巻発電機,マグネト発電機,複巻発電機がある.\\
 この後,出てくるパラメータを下に示す.
\begin{subnumcases}
  {}
  V:\mbox{端子電圧}[\textrm{V}]\nonumber \\
  E:\mbox{起電力}[\textrm{V}]\nonumber \\
  R_a:\mbox{電機子巻線抵抗}[\Omega]\nonumber \\
  R_s:\mbox{界磁巻線抵抗}[\Omega]\nonumber \\
  I:\mbox{負荷電流}[\textrm{A}]\nonumber \\
  V_b:\mbox{ブラシの電圧降下}[\textrm{V}]\nonumber \\
  V_c:\mbox{電機子反作用による電圧降下}[\textrm{V}]\nonumber \\
  P:\mbox{出力}[\textrm{W}]\nonumber \\
  P':\mbox{発生電力}[\textrm{VA}]\nonumber \\
  I_f:\mbox{界磁電流}[\textrm{A}]\nonumber \\
  R_f:\mbox{界磁抵抗}[\Omega]\nonumber
\end{subnumcases}

\subsubsection{他励発電機}
磁界を発生させる界磁巻線の電源が別電源.
\pic{imgs/dummy.png}
\begin{eqnarray}
  V &=& E-R_a I - (V_b + V_c)\, [\textrm{V}] \\
  P &=& VI \\
  P' &=& EI
\end{eqnarray}

\subsubsection{分巻発電機}
自励発電機であり,界磁巻線が並列の発電機

  \pic{imgs/dummy.png}

\begin{eqnarray}
  V &=& E-R_a (I+I_f) - (V_b + V_c)\, [\textrm{V}] \\
  P &=& VI \\
  P' &=& E(I+I_f) \\
  I_f &=&\frac{V}{R_f}\, [\textrm{A}]
\end{eqnarray}

\subsubsection{直巻発電機}
自励発電機であり,界磁巻線が直列の発電機

  \pic{imgs/dummy.png}

\begin{eqnarray}
  V &=& E-(R_a+R_s)I - (V_b + V_c)\, [\textrm{V}] \\
  P &=& VI \\
  P' &=& EI
\end{eqnarray}

\subsubsection{複巻発電機}
\begin{itemize}
  \item 自励発電機であり,界磁巻線が分巻と直巻の発電機
  \item 和動複巻直流発電機:磁気を強め合う極性
  \item 差動複巻直流発電機:磁気を相殺し合う極性
\end{itemize}

\subsubsection{マグネト発電機}
\begin{itemize}
  \item 永久磁石を用いた発電機
\end{itemize}

\subsection{発電機の特性}
\begin{itemize}
  \item 無負荷飽和曲線:回転速度一定で無負荷状態の界磁電流と誘導起電力の関係を示す
  \item 外部特性曲線 :負荷電流に対する端子電圧の関係を示す
  \item 無負荷飽和曲線は分巻も基本は同様.
  \item 直巻は「負荷電流=界磁電流」のため.無負荷飽和曲線はない.
\end{itemize}


  \pic{imgs/dummy.png}

