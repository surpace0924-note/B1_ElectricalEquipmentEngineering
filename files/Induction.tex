\section{誘導機}
\subsection{誘導機}

\begin{classify}{\fbox{誘導機}}
  \classf{発電機}
\class
{
  \begin{classify}{\fbox{電動機}}
    \classf{三相}
    \classf{単相}
\end{classify}
}
\end{classify}

\begin{itemize}
  \item 固定子の作る回転磁界により,電気伝導体の回転子に誘導
  \item 電流が発生し,滑りに対応した回転トルクが発生.
  \item 堅牢,安価,取り扱いが容易.
  \item 一般に「誘導機→三相交流誘導電動機」
  \item 回転子導体の種類により,かご形と巻線形に分類.
  \item 交流電源の種類により,単相と三相に大別.
\end{itemize}

\subsection{回転子導体の構造による分類}
\subsubsection{かご形誘導電動機}
\begin{itemize}
  \item トルク制御と励磁制御を分離したベクトル制御方式は,直流電動機のような速度制御が可能.
  \item 構造が簡単,堅牢,安価
\end{itemize}

\subsubsection{巻線形誘導電動機}
\begin{itemize}
  \item スリップリングが接続されている.
  \item ブラシはスリップリングに接続されている.
\end{itemize}

\subsection{回転磁界}
\begin{itemize}
  \item S・N極が中点もしくは,ある軸を中心に回転しているかのように極性が変化する磁界.
\end{itemize}

\subsection{同期速度}
\begin{eqnarray}
  n_s &=& \frac{f}{p}\, [\textrm{rps}] \\
  N_s &=& \frac{2f}{P} \cdot 60\, [\textrm{rpm}]
\end{eqnarray}
\begin{subnumcases}
  {}
  n_s:\mbox{回転磁界の回転数}[\textrm{rps}]\nonumber \\
  N_s:\mbox{同期速度}[\textrm{rpm}]\nonumber \\
  f:\mbox{周波数}[\textrm{Hz}]\nonumber \\
  p:\mbox{対極数}\nonumber \\
  P:\mbox{極数}(=2p)\nonumber
\end{subnumcases}

\subsection{すべり}
\begin{itemize}
  \item 相対速度と同期速度との比
  \item 相対速度は同期速度$N_s$と実際の回転数$N$との差
\end{itemize}
\begin{eqnarray}
  S &=& \frac{N_s - N}{N_s}\\
  N &=& N_s(1-S) \, [\textrm{rpm}]
\end{eqnarray}
\begin{subnumcases}
  {}
  S:\mbox{すべり}\nonumber \\
  N_s:\mbox{同期速度}[\textrm{rpm}]\nonumber \\
  N:\mbox{実際の回転数}[\textrm{rpm}]\nonumber
\end{subnumcases}

\subsubsection{2次周波数}
\begin{equation}
  f_{2s} = S \cdot f_1\, [\textrm{Hz}]
\end{equation}

\subsubsection{一相の2次誘導起電力}
\begin{equation}
  E_{2s} = S \cdot E_2\, [\textrm{V}]
\end{equation}

\subsubsection{一相の1次誘導起電力}
\begin{equation}
  E_1 : E_{2s} = 1 : S
\end{equation}

\subsection{電力}
\begin{equation}
  \mbox{効率} \eta = \frac{P_m}{P_1}
\end{equation}
\begin{equation}
  \mbox{2次入力} : \mbox{2次出力} : \mbox{2次銅損} = 1: 1-S : S
\end{equation}

\subsection{トルク}
\begin{eqnarray}
  T &=& \frac{P_m}{\omega} [\textrm{Nm}]\\
  \omega &=& 2 \pi N \, [\textrm{rad/s}]
\end{eqnarray}
