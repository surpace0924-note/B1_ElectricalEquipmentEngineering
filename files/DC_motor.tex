\section{直流電動機}
\subsection{直流電動機}
直流電動機は直流を入力とする電動機.構造は直流発電機と同じ.

  \pic{imgs/dummy.png}


  \pic{imgs/dummy.png}


\subsection{逆起電力}
発電機と同様に起電力が発生し,電機子電流を妨げるように働く.\\

 (\ref{DC_Electromotiveforce})式と同じ
\begin{equation}
  \label{DC_E_2}
  E =  \frac{Z}{a} \cdot p \phi n  = K \cdot \phi n \, [\textrm{V}]
\end{equation}
この時,電機子電流$I_a$は
\begin{equation}
  \label{DC_Ia}
  I_a = \frac{V-E}{R_a}\, [\textrm{A}]
\end{equation}
\begin{subnumcases}
  {}
  V:\mbox{端子電圧}[\textrm{V}]\nonumber \\
  E:\mbox{起電力}[\textrm{V}]\nonumber \\
  R_a:\mbox{電機子巻線抵抗}[\Omega]\nonumber \\
  I_a:\mbox{電機子電流}[\textrm{A}]\nonumber
\end{subnumcases}
端子電圧と内部電圧降下との関係は(\ref{DC_Ia})式を変形して
\begin{equation}
  \label{DC_V}
  V = E + R_a \cdot I_a\, [\textrm{V}]
\end{equation}

\subsection{速度}
直流電動機の回転速度$n[\textrm{rps}]$は逆起電力に比例し,1極あたりの磁束に反比例する.
(\ref{DC_E_2})式を変形して
\begin{eqnarray}
  n &=& \frac{E \cdot a}{p \cdot \phi \cdot Z}\, [\textrm{rps}] \\
  n &=& K \cdot \frac{E}{\phi}\, [\textrm{rps}] \\
  n &=& K \cdot \frac{V - (R_a \cdot I_a)}{\phi}\, [\textrm{rpm}]
\end{eqnarray}

\subsection{トルク}
導体1本に加わる力$f$は
\begin{equation}
  \label{DC_F}
  f = Bl \cdot \frac{I_a}{a}\, [\textrm{N}]
\end{equation}

導体1本のトルク$t$は以下の式で求まる(電機子の半径を$r$とする)
\begin{equation}
  \label{DC_T_basic}
  t = f \cdot r = \frac{BlI_ar}{a}\, [\textrm{Nm}]
\end{equation}

ここで$B$は
\begin{equation}
  \label{DC_B}
  B = \frac{\phi}{\frac{2 \pi r l}{p}} = \frac{p \phi}{2 \pi r l}\, [\textrm{T}]
\end{equation}

(\ref{DC_T_basic})式と(\ref{DC_B})式より
\begin{equation}
  \label{T_basic}
  t = \frac{p \phi I_a}{2 \pi a}\, [\textrm{Nm}]
\end{equation}

最後に全導体数$Z$をかけて全体のトルク$T$を出す
\begin{equation}
  \label{T_basic}
  T = \frac{p \phi I_a}{2 \pi a} \cdot Z = K_2 \cdot \phi I_a\, [\textrm{Nm}]
\end{equation}

\subsection{出力}
(\ref{DC_V})の両端に$I_a$をかけると
\begin{equation}
  VI_a = EI_a + R_a{I_a}^2\, [\textrm{W}]
\end{equation}

機械的動力に変換される電力$P_m$は
\begin{equation}
  P_m = EI_a = 2 \pi nT\, [\textrm{W}]
\end{equation}

電動機の出力は
\begin{equation}
  P = P_m - (\mbox{鉄損} + \mbox{機械的諸損失})\, [\textrm{W}]
\end{equation}

\subsection{電圧変動率}
電動機を定格電圧,定格出力で運転して,定格負荷から無負荷にした時の電圧変動の割合.
\begin{equation}
  \epsilon = \frac{V_0 - V_n}{V_n} \cdot 100\, [\textrm{\%}]
\end{equation}
\begin{subnumcases}
  {}
  \epsilon:\mbox{電圧変動率}[\textrm{V}]\nonumber \\
  V_0:\mbox{無負荷電圧}[\textrm{V}]\nonumber \\
  V_n:\mbox{定格電圧}[\Omega]\nonumber
\end{subnumcases}

\subsection{電機子反作用}
電動機の電気的中性軸は,発電機の反対方向へ移動する.

\subsection{始動}
\begin{itemize}
  \item 静止状態の電動機を運転状態に移すこと.
  \item 始動器を用いることがある.
\end{itemize}

\subsection{損失}
\begin{itemize}
  \item 一部のエネルギーは熱となって失われ,損失となる.
  \item 機械損,鉄損,銅損(抵抗損),浮遊負荷損など.
\end{itemize}

\subsection{効率}
\begin{itemize}
  \item 実測効率と規約効率の二つ.
  \item 一般的には規約効率が用いられる.
\end{itemize}

\begin{eqnarray}
  \mbox{実測効率} &=& \frac{\mbox{出力}}{\mbox{入力}} \cdot 100\, [\textrm{\%}] \\
  \mbox{規約効率(発電機)} &=& \frac{\mbox{出力}}{\mbox{出力} + \mbox{損失}} \cdot 100\, [\textrm{\%}] \\
  \mbox{規約効率(電動機)} &=& \frac{\mbox{入力} - \mbox{損失}}{\mbox{入力}} \cdot 100\, [\textrm{\%}]
\end{eqnarray}

\subsection{定格}
\begin{itemize}
  \item 出力の限度.
  \item その他定格速度,定格電圧,定格電流 etc.
  \item 定格には$n$をつけることがほとんど.$Pn$,$Vn$ etc.
\end{itemize}
