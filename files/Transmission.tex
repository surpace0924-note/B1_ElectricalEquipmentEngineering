\section{送配電}
\subsection{送電電圧}
\begin{itemize}
  \item 発電所で発電した電力は送電・配電線によって需要家に送られる.
  \item 送電電圧は高圧が6.6kVそのほかに特別高圧・超高圧(11〜275kV),超々高圧(500kV)などが決められている.
\end{itemize}

\subsection{送電の流れ}
\begin{enumerate}
  \item 発電された電圧は154〜500kVに昇圧され,一次送電線(基幹送電線)で送電する.
  \item 一次変電所から二次変電所や特別高圧需要家に送られる.
  \item 二次送電線を介して,一般の事業所や工場,家庭などに配電される.
\end{enumerate}

\subsection{周波数}
\begin{itemize}
  \item 東日本 → 50Hz(ドイツ製)
  \item 西日本 → 60Hz(アメリカ製)
  \item 周波数変換所により,東西で融通できる
\end{itemize}

\subsection{力率}
\begin{itemize}
  \item 以下,電圧実効値を$V$,電流実効値を$I$,有効電力を$P$.
  \item 皮相電力$S=VI$
  \item 力率$=P$/$S$$=cos\phi$
  \item 有効電力$P=VIcos\phi$
  \item 位相差が生じる場合(主に誘導性負荷),力率改善のために進相コンデンサを挿入する.
\end{itemize}

\subsection{三相回路}

  \pic{imgs/dummy.png}

\begin{itemize}
  \item 三相回路のO点は中性点と呼ばれ,0Vになる.
  \item Y形の接続をスター結線という.
  \item $E_a$,$E_b$,$E_c$  → 相電圧
  \item $V_ab$,$V_bc$,$V_ca$ → 線間電圧
\end{itemize}

\subsection{送電線}
\begin{itemize}
  \item 架空送電線と地中送電線がある.
  \item 電力が大きく,距離が大きいほど,高電圧で送電する方が線路損失が小さくなる.
\end{itemize}

\subsubsection{架空送電線}
\begin{itemize}
  \item 送電用鉄塔,電力線,架空地線,がいしなどで構成.
  \item 電力線 :鋼心アルミ撚り線が用いられる.
  \item がいし :電力線と送電鉄塔とを電気的に絶縁する.
  \item 架空地線:落雷などによる電力線への雷撃を防ぐもの(接地されていて,送電には用いない).
\end{itemize}

  \pic{imgs/dummy.png}


\subsubsection{地中送電線}
\begin{itemize}
  \item 国内の都市や郊外で使用されるようになってきた.
  \item メリット :用地が少ない,美観,落雷の心配がない
  \item デメリット:建設費用が高い.送電容量が小さい.復旧に時間がかかる.
  \item ケーブル :OF(oil filled)ケーブル,CVケーブルが主に使われる.
\end{itemize}

\subsection{変電所の機器}
\begin{itemize}
  \item 変電所には変圧器,開閉装置,保護装置,調相設備などがある.
\end{itemize}

\subsubsection{変圧器}
\begin{itemize}
  \item 昇圧や降圧を行う.
\end{itemize}

\subsubsection{開閉装置}
\begin{itemize}
  \item 遮断器と断路器を指す.
  \item 遮断器:電力系統から切り離す.
  \item 断路器:遮断器が切れている状態で各種機器と開閉する.
\end{itemize}

\subsubsection{保護装置}
\begin{itemize}
  \item 主に避電器を指す.
  \item 電力系統に発生する雷サージや開閉サージから保護する
\end{itemize}

\subsubsection{GIS}
\begin{itemize}
  \item ガス絶縁開閉装置(Gas Insulated Switchgear)の略
  \item 遮・断・避・開などを収納し,$SF_6$ガスを充満した装置.
\end{itemize}

\subsubsection{調相設備}
\begin{itemize}
  \item 負荷の力率改善や無効電力調整を行う.
\end{itemize}

\subsection{配電システム}
\begin{itemize}
  \item 高圧配電線から柱上変圧器で低圧にして,一般家庭へ配電.
  \item 線間電圧によって以下のように区分.
  \item 低圧  :600V以下
  \item 高圧  :7kV以上
  \item 特別高圧:それ以上
\end{itemize}

\subsection{低圧配電系統}
\begin{itemize}
  \item 単相2線式:コンセント用 (100 or 200V)
  \item 単相3線式:中性線入り  (100 or 200V)
  \item 三相3線式:Δ結線を用いた(200V)
\end{itemize}

\subsection{負荷の特性}
\begin{itemize}
  \item 負荷曲線:負荷の変動を時間的に表した曲線
  \item 需要率 :最大需要電力の設備容量に対する比率
\end{itemize}
\begin{eqnarray}
  \mbox{負荷率} &=& \frac{\mbox{平均負荷電力}}{\mbox{最大負荷電力}} \cdot 100\, [\textrm{\%}]\\
  \mbox{需要率} &=& \frac{\mbox{最大需要電力}}{\mbox{設備容量}} \cdot 100\, [\textrm{\%}]
\end{eqnarray}

\subsection{電圧降下}
\begin{itemize}
  \item 受電端電圧 = 送電端電圧 - 電圧降下
  \item 国内では100Vは101±6V,200Vは202±20Vの範囲内に維持されるように電圧制御が行われている.
\end{itemize}

\subsection{力率調整}
\begin{itemize}
  \item 損失低減のために,受電端に電力用コンデンサを挿入して,負荷力率を改善する.
\end{itemize}
\begin{equation}
  \mbox{基本料金} = (\mbox{契約基本料金}) \cdot (1 + \frac{85 - \mbox{力率}}{100})
\end{equation}

\subsection{電力品質}
\begin{itemize}
  \item 気候で変動するエネルギーにより,電力品質の低下する.
  \item 電力品質のパラメータとして,高調波や瞬時電圧降下がある.
  \item 対策として,無停電電源装置(UPS)が有効
\end{itemize}

\subsection{直流送電}
\begin{itemize}
  \item 交流への変換装置が必要 → 経済的に不利
  \item 遮断が難しい.
  \item 長距離大電力送電に適してる.
  \item 送電経路が「+」「-」だけのため,建設費が安価.
  \item 電圧降下や電力損失が少ない.
\end{itemize}

\subsection{交流送電}
\begin{itemize}
  \item 変圧器が使える.
  \item 回転機のメンテナンスが容易.
\end{itemize}

\subsection{交直変換所}
\begin{itemize}
  \item 交流と直流を変換する変換所
  \item 交直変換装置,サイリスタバルブなど
\end{itemize}

\subsubsection{交直変換装置}
\begin{itemize}
  \item AC → DC:コンバータ
  \item DC → AC:インバータ
\end{itemize}

\subsubsection{サイリスタバルブ}
\begin{itemize}
  \item 高電圧,大容量のサイリスタ素子を多数個組み合わせたもの.
  \item 整流器の役割.
\end{itemize}
